%%%%%%%%%%%%%%%%%%%%%%%%%%%%%%%%%%%%%%%%%
% Short Sectioned Assignment
% LaTeX Template
% Version 1.0 (5/5/12)
%
% This template has been downloaded from:
% http://www.LaTeXTemplates.com
%
% Original author:
% Frits Wenneker (http://www.howtotex.com)
%
% License:
% CC BY-NC-SA 3.0 (http://creativecommons.org/licenses/by-nc-sa/3.0/)
%
%%%%%%%%%%%%%%%%%%%%%%%%%%%%%%%%%%%%%%%%%

%----------------------------------------------------------------------------------------
% PACKAGES AND OTHER DOCUMENT CONFIGURATIONS
%----------------------------------------------------------------------------------------

\documentclass[paper=a4, fontsize=11pt]{scrartcl} % A4 paper and 11pt font size

\usepackage[T1]{fontenc} % Use 8-bit encoding that has 256 glyphs
\usepackage{fourier} % Use the Adobe Utopia font for the document - comment this line to return to the LaTeX default
\usepackage[english]{babel} 
\usepackage[utf8]{inputenc}
\usepackage{amsmath,amsfonts,amsthm} % Math packages
\usepackage{tikz} % drawing package

\usepackage{sectsty} % Allows customizing section commands
\allsectionsfont{\centering \normalfont\scshape} % Make all sections centered, the default font and small caps

\usepackage{fancyhdr} % Custom headers and footers
\pagestyle{fancyplain} % Makes all pages in the document conform to the custom headers and footers
\fancyhead{} % No page header - if you want one, create it in the same way as the footers below
\fancyfoot[L]{} % Empty left footer
\fancyfoot[C]{} % Empty center footer
\fancyfoot[R]{\thepage} % Page numbering for right footer
\renewcommand{\headrulewidth}{0pt} % Remove header underlines
\renewcommand{\footrulewidth}{0pt} % Remove footer underlines
\setlength{\headheight}{13.6pt} % Customize the height of the header

\numberwithin{equation}{section} % Number equations within sections (i.e. 1.1, 1.2, 2.1, 2.2 instead of 1, 2, 3, 4)
\numberwithin{figure}{section} % Number figures within sections (i.e. 1.1, 1.2, 2.1, 2.2 instead of 1, 2, 3, 4)
\numberwithin{table}{section} % Number tables within sections (i.e. 1.1, 1.2, 2.1, 2.2 instead of 1, 2, 3, 4)

\setlength\parindent{0pt} % Removes all indentation from paragraphs - comment this line for an assignment with lots of text

%---- Listings --------------
\usepackage{color}
\definecolor{light-gray}{gray}{0.95}
\usepackage{listings}

\lstnewenvironment{code}[1][]%
{\minipage{\linewidth}
\lstset{ %
language={[x86masm]Assembler},  % choose the language of the code
basicstyle=\footnotesize,       % the size of the fonts that are used for the code
numbers=left,                   % where to put the line-numbers
numberstyle=\footnotesize,      % the size of the fonts that are used for the line-numbers
stepnumber=1,                   % the step between two line-numbers. If it is 1 each line will be numbered
resetmargins=true,              % reset line numbers
numbersep=5pt,                  % how far the line-numbers are from the code
backgroundcolor=\color{white},  % choose the background color. You must add \usepackage{color}
showspaces=false,               % show spaces adding particular underscores
showstringspaces=false,         % underline spaces within strings
showtabs=false,                 % show tabs within strings adding particular underscores
frame=single,                   % adds a frame around the code
tabsize=2,                      % sets default tabsize to 2 spaces
captionpos=b,                   % sets the caption-position to bottom
breaklines=true,                % sets automatic line breaking
breakatwhitespace=false,        % sets if automatic breaks should only happen at whitespace
escapeinside={\%*}{*)},         % if you want to add a comment within your code
morekeywords={
    orr, ldr, bne, subs
    },                          % additional keywords to expand the asm language
#1
}}%
{\endminipage}

%----------------------------------------------------------------------------------------
% TITLE SECTION
%----------------------------------------------------------------------------------------

\newcommand{\horrule}[1]{\rule{\linewidth}{#1}} % Create horizontal rule command with 1 argument of height

\title{
\normalfont \normalsize
\textsc{TDT4205 Compilers, NTNU} \\ [25pt] % Your university, school and/or department name(s)
\horrule{0.5pt} \\[0.4cm] % Thin top horizontal rule
\huge Problem Set 5 \\ % The assignment title
\horrule{2pt} \\[0.5cm] % Thick bottom horizontal rule
}

\author{Stian Jensen} % Your name

\date{\normalsize 12. March 2014} % Today's date or a custom date

\begin{document}

\maketitle % Print the title

%----------------------------------------------------------------------------------------
% PROBLEM 1
%----------------------------------------------------------------------------------------

\section{Type checking}

In addition to the type checking already implemented, one could typecheck return statements, to check that they match the defined return type of their function or method.
We could also typecheck expressions used in if/while statements, to check that they are boolean.
However, in these cases it is often convention to let many datatypes evaluate to either true or false. For instance, in many languages "if (i+1) then ..." would be valid, even though the expression (i+1) is not boolean.

\newpage

\section{Assembly programming}

\begin{table}[ht!]
    \begin{tabular}{lll}
    Pointers & Address     & Stack          \\
    \hline
    ~        & ...         & ...            \\
    ~        & ~           & Return Address \\
    FP       & (main\_fp)  & Saved FP       \\
    ~        & (main\_fp)-4 & a              \\
    ~        & (main\_fp)-8 & b              \\
    ~        & (f\_fp)+8   & 5              \\
    ~        & (f\_fp)+4   & Return Address \\
    FP       & (f\_fp)     & Saved FP       \\
    ~        & (f\_fp)-4   & b              \\
    ~        & (f\_fp)-8   & c              \\
    ~        & (f\_fp)-12  & d              \\
    ~        & (g\_fp)+16  & b              \\
    ~        & (g\_fp)+12  & c              \\
    ~        & (g\_fp)+8   & d              \\
    ~        & (g\_fp)+4   & Return Address \\
    FP       & (g\_fp)     & Saved FP       \\
    ~        & (g\_fp)-4   & e              \\
    SP       & (g\_fp)-8   & d              \\
    \end{tabular}
    \caption{The stack at position 1}
\end{table}

\section{Assembly programming}

\begin{code}

_f:
    push {lr}
    push {fp}
    mov fp, sp

    push {#0}

    ldr r1, [fp, #12]
    str r1, [fp, #-4]

    ldr r1, [fp, #8]
    push {r1}
    ldr r2, [fp, #-4]
    push {r2}
    bl _g
    pop
    pop

    mov sp, fp
    pop {fp}
    pop {pc}

_g:
    push {lr}
    push {fp}
    mov fp, sp
    ldr r0, [fp, #8]
    mov sp, fp
    pop {fp}
    pop {pc}

\end{code}

\end{document}
